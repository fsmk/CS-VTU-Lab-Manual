\documentclass[a4paper]{report}
\usepackage{fancyvrb,graphicx,color,fancyhdr,enumerate,url}
\usepackage[margin=1.8cm]{geometry}
\usepackage[svgnames]{xcolor} % Required to specify font color
\pagestyle{fancy}
\fancyfoot{}
\definecolor{light-gray}{gray}{0.95}
%----------------------------------------------------------------------------------------
%	TITLE PAGE
%----------------------------------------------------------------------------------------

\thispagestyle{empty}
\newcommand*{\rotrt}[1]{\rotatebox{90}{#1}} % Command to rotate right 90 degrees
\newcommand*{\rotlft}[1]{\rotatebox{-90}{#1}} % Command to rotate left 90 degrees

\newcommand*{\titleBC}{\begingroup % Create the command for including the title page in the document
\centering % Center all text

\includegraphics[scale=.8]{sitlogo.png}

\Large {SIDDAGANGA INSTITUTE OF TECHNOLOGY \\ www.sit.ac.in} % Year published


\def\CP{\textit{\LARGE Foundations of Computer Programming Lab}} % Title

\settowidth{\unitlength}{\CP} % Set the width of the curly brackets to the width of the title
{\color{Orange}\resizebox*{\unitlength}{\baselineskip}{\rotrt{$\}$}}} \\[\baselineskip] % Print top curly bracket
\textcolor{Sienna}{\CP} \\[\baselineskip] % Print title
{\color{RosyBrown}\Large LAB MANUAL} \\ % Tagline or further description
{\color{Orange}\resizebox*{\unitlength}{\baselineskip}{\rotlft{$\}$}}} % Print bottom curly bracket

\vfill % Whitespace between the title and the author name

{\large Prabodh C P\\Asst Professor \\ Dept of CSE, SIT }\\ % Author name

\vfill % Whitespace between the author name and the publisher logo

\rule{\textwidth}{0.4pt}\vspace*{-\baselineskip}\vspace{3.2pt} % Thin horizontal line
\rule{\textwidth}{1.6pt}\\[\baselineskip] % Thick horizontal line

\includegraphics[scale=0.25]{gnu-gcc-cb.png} \\

\rule{\textwidth}{1.6pt}\vspace*{-\baselineskip}\vspace*{2pt} % Thick horizontal line
\rule{\textwidth}{0.4pt}\\[\baselineskip] % Thin horizontal line
\includegraphics[scale=1]{cc-by-sa.png} \\
 \begin{center}
\small{This work is licensed under a Creative Commons Attribution-ShareAlike 4.0 International License.}
\end{center}
\endgroup}

%----------------------------------------------------------------------------------------
%	BLANK DOCUMENT
%----------------------------------------------------------------------------------------


%\usepackage{pslatex,palatino,avant,graphicx,color}

%to restart chapter numbering after Part
    \makeatletter
    \@addtoreset{chapter}{part}
    \makeatother

\pagestyle{fancy}
\fancyfoot[L]{Dept of CSE, SIT}
\fancyfoot[R]{\thepage}
%\lfoot[]{Dept of CSE, SIT}
%\rfoot{\thepage}    
\begin{document}
%\pagestyle{empty} % Removes page numbers

\newpage
\pagecolor{Goldenrod!50}
\titleBC % This command includes the title page
\tableofcontents
\newpage
\pagecolor{white}
\chapter*{•}
\section*{\begin{center}
FCP LABORATORY\\Common to all Branches
\end{center}}
\section*{Instructions}
\begin{itemize}
\item All the C programs need to be executed on Codeblocks IDE using GCC Compiler.
\item Algorithms and Flowcharts are compulsory for all the programs.
\item Demo experiments are only for assignment purpose, not to be included in examinations.
\item All experiments must be included in practical examinations. 
\item Students are allowed to pick one experiment each from Part A and Part B from the lots. 
\end{itemize}

\section*{Demo Experiments}
\begin{enumerate}
\item Demonstrate and explain the organization of a Personal Computer and its accessories. 
\item Explanation of Concepts – Computer Software, Computing Environment, Computer Languages, Creating and running programs
\end{enumerate}
Students have to prepare a write-up for the above demonstrations and include it in the Lab record for evaluation. 

\section*{References}
\textbf{Part A}: Behrouz A. Forouzan , Richard F. Gilberg , Computer Science: \textbf{A Structured programming Approach Using C} - Cengage Learning; 3rd edition \\
For writing flowcharts refer to \textbf{Appendix C} of the above book.

\part{PART A}
\chapter{Sales of a shop}
{\fontfamily{ppl}\selectfont \textbf{Write a program to calculate the total sales of a shop, given the unit price, quantity, discount rate and sales tax rate of a particular product. The sales tax is 8.5\% which should be specified using a defined constant. The output should be displayed in proper form using appropriate format specifiers. \\
}}

\section*{C Code}

\begin{Verbatim}
#include <stdio.h>
#include <stdlib.h>

#define SALES_TAX 8.5

int main(void)
{
    int iQuantity;
    float fDiscRate, fDiscAmount, fUnitPrice, fSubTotal;
    float fSubTaxable, fTaxAm, fTotal;
    printf("\nEnter the number of items sold...\t:");
    scanf("%d",&iQuantity);

    printf("\nEnter the unit price... \t\t:");
    scanf("%f",&fUnitPrice);

    printf("\nEnter the discount rate (%%)... \t\t:");
    scanf("%f",&fDiscRate);

    fSubTotal = iQuantity * fUnitPrice;
    fDiscAmount = fSubTotal * fDiscRate / 100.0;
    fSubTaxable = fSubTotal - fDiscAmount;
    fTaxAm = fSubTaxable * SALES_TAX / 100.0;
    fTotal = fSubTaxable + fTaxAm;

    printf("\nQuantity sold:    \t%3d\n", iQuantity);
    printf("Unit Price of items: %9.2f\n",fUnitPrice);
    printf("                    ----------------\n");

    printf("SubTotal :           %9.2f\n", fSubTotal);
    printf("Discount :           -%8.2f\n", fDiscAmount);
    printf("Discounted total :   %9.2f\n", fSubTaxable);
    printf("Sales Tax :          +%8.2f\n", fTaxAm);
    printf("Total sales :        %9.2f\n", fTotal);

    return 0;
}
\end{Verbatim}

\section*{Output}
\begin{Verbatim}


Enter the number of items sold...	:35

Enter the unit price... 		 :15.55

Enter the discount rate (%)... 	  :6

Quantity sold:    	 35
Unit Price of items:     15.55
                    ----------------
SubTotal :              544.25
Discount :           -   32.65
Discounted total :      511.60
Sales Tax :          +   43.49
Total sales :           555.08



Enter the number of items sold...	:43

Enter the unit price... 		 :22.75

Enter the discount rate (%)... 	  :8

Quantity sold:    	 43
Unit Price of items:     22.75
                    ----------------
SubTotal :              978.25
Discount :           -   78.26
Discounted total :      899.99
Sales Tax :          +   76.50
Total sales :           976.49


\end{Verbatim}


\chapter{Sum of last two digits}
{\fontfamily{ppl}\selectfont \textbf{Write a program to extract and add the two least significant digits of an integer using the following user defined functions.\\
\begin{enumerate}[i]
\item Function that obtains the unit place digit of a given number
\item Function that obtains the  tenth place digit of a given number
\item Function that adds unit and tenth place digits of a given number
\end{enumerate}
	The main function should call these functions. 
}}
\section*{C Code}
\begin{Verbatim}
/***************************************************************************
*File           : SumTwoDigits.c
*Descriptio     : Program to add the last two digits of a given number
*Author         : Prabodh C P
*Compiler       : gcc compiler, Ubuntu 14.04
*Date           : 4 August 2014
***************************************************************************/
#include <stdio.h>
#include <stdlib.h>
int fnExtractUnit(int);
int fnExtractTen(int);
int fnAddUnitTen(int, int);

int main(void)
{
    int iNum,iTen,iUnit,iSum;

    printf("\nEnter a number\n");
    scanf("%d", &iNum);

    iUnit = fnExtractUnit(iNum);
    iTen = fnExtractTen(iNum);
    iSum = fnAddUnitTen(iTen, iUnit);

    printf("\nThe unit digit in %d is %d\n",iNum, iUnit);
    printf("\nThe ten's digit in %d is %d\n",iNum, iTen);
    printf("\nThe sum of last two digits in %d is %d\n",iNum, iSum);

    return 0;
}

int fnExtractUnit(int iVal)
{
    int iU;
    iU = (iVal%10);
    return iU;
}

int fnExtractTen(int iVal)
{
    int iT;
    iT = (iVal%100)/10;
    return iT;
}

int fnAddUnitTen(int iVal1, int iVal2)
{
    int iS;
    iS =  (iVal1 + iVal2);
    return iS;
}

\end{Verbatim}
\section*{Output}
\begin{Verbatim}


Enter a number
4567

The unit digit in 4567 is 7

The ten's digit in 4567 is 6

The sum of last two digits in 4567 is 13


Enter a number
1209

The unit digit in 1209 is 9

The ten's digit in 1209 is 0

The sum of last two digits in 1209 is 9


Enter a number
76

The unit digit in 76 is 6

The ten's digit in 76 is 7

The sum of last two digits in 76 is 13


\end{Verbatim}

\chapter{Boiling Points}
{\fontfamily{ppl}\selectfont \textbf{The following table shows the normal boiling points of several substances. Write a program that reads a boiling point of a substance and identifies the substance if the input boiling point is within 5\% of the expected boiling point \\ \\
\begin{tabular}{l c r}
	Substance	& Normal Boiling Point \\
	Water 		& 100 \\
	Mercury 	& 357 \\
	Copper 		& 1187 \\
	Silver 		& 2193 \\
	Gold 		& 2660 \\
\end{tabular}
\\ \\For example if the input is 104 it lies in the 5\% range of 100 which is from 95 to 105. Hence the substance has to be water 
}}
\section*{C Code}
\begin{Verbatim}
#include <stdio.h>
#include <stdlib.h>

int main(void)
{
    float fBoilPt;

    printf("\nEnter Boiling point of the substance\n");
    scanf("%f",&fBoilPt);

    if (0.95*100 <= fBoilPt && fBoilPt <= 1.05*100)
        printf("\nThe substance is Water\n");
    else if (0.95*357 <= fBoilPt && fBoilPt <= 1.05*357)
        printf("\nThe substance is Mercury\n");
    else if (0.95*1187 <= fBoilPt && fBoilPt <= 1.05*1187)
        printf("\nThe substance is Copper\n");
    else if (0.95*2193 <= fBoilPt && fBoilPt <= 1.05*2193)
        printf("\nThe substance is Silver\n");
    else if (0.95*2660 <= fBoilPt && fBoilPt <= 1.05*2660)
        printf("\nThe substance is Gold\n");
    else
        printf("\nInvalid Substance\n");
    return 0;
}

\end{Verbatim}

\pagebreak
\section*{Output}
\begin{Verbatim}


Enter Boiling point of the substance
102

The substance is Water


Enter Boiling point of the substance
365

The substance is Mercury


Enter Boiling point of the substance
1200

The substance is Copper


Enter Boiling point of the substance
2205

The substance is Silver


Enter Boiling point of the substance
2670

The substance is Gold


Enter Boiling point of the substance
3333

Invalid Substance

\end{Verbatim}

\chapter{Roots of a Quadratic Equation}
{\fontfamily{ppl}\selectfont \textbf{Write a program using switch statement, to find all the possible roots of Quadratic equation. Display suitable error messages for invalid inputs.  Write a user defined function \textit{fnCalcDiscriminant} to calculate the discriminant .
}}
\section*{Summary}
Any quadratic equation has two roots and the roots of the equation can be found using the formula
\[x=\frac{-b\pm\sqrt{discriminant}}{2*a}\] where \(discriminant={b^{2}-4*a*c}\) and a,b and c are the coefficients of the quadratic equation \textbf{\(ax^{2}+bx+c=0\)}\\
Hence as per the equation, if discriminant is equal to 0, then the value of both the roots are equal and real. Also to find x, when discriminant is not 0, we need to find square root of disc. When discriminant value is less than 0, the square root of the discriminant is imaginary and hence the roots of the equation are supposed to be imaginary and distinct. When discriminant value is greater than 0, the square root of the discriminant is real and hence the roots of the equation are supposed to be real and distinct.


\section*{C Code}
\begin{Verbatim}
/***************************************************************************
*File		: Quadratic.c
*Description	: Program to find the roots of a Quadratic Equation
*Author		: Prabodh C P
*Compiler	: gcc compiler, Ubuntu 14.04
*Date		: 3 August 2014
***************************************************************************/

#include<stdio.h>
#include<stdlib.h>
#include<math.h>
float fnCalcDiscriminant(float, float, float);
/***************************************************************************
*Function	: 	main
*Input parameters:	no parameters
*RETURNS	:	0 on success
***************************************************************************/

int main(void)
{
	float fA,fB,fC,fDesc,fX1,fX2,fRealp,fImagp;
	int iState;

	printf("\n*************************************************************");
	printf("\n*\tPROGRAM TO FIND ROOTS OF A QUADRATIC EQUATION\t    *\n");
	printf("*************************************************************");


	printf("\nEnter the coefficients of a,b,c \n");
	scanf("%f%f%f",&fA,&fB,&fC);
	if(0 == fA)
	{
		printf("\nInvalid input, not a quadratic equation - try again\n");
		exit(0);
	}

	/*COMPUTE THE DESCRIMINANT*/
	fDesc=fnCalcDiscriminant(fA,fB,fC);
	((0 == fDesc) ? (iState = 1):((fDesc > 0) ? (iState = 2) : (iState = 3)));
	switch(iState)
	{
		case 1:
			fX1 = fX2 = -fB/(2*fA);
			printf("\nRoots are equal and the Roots are \n");
			printf("\nRoot1 = %g and Root2 = %g\n",fX1,fX2);
			break;
		case 2:
			fX1 = (-fB+sqrt(fDesc))/(2*fA);
			fX2 = (-fB-sqrt(fDesc))/(2*fA);
			printf("\nThe Roots are Real and distinct, they are \n");
			printf("\nRoot1 = %g and Root2 = %g\n",fX1,fX2);
			break;
		case 3:
			fRealp = -fB / (2*fA);
			fImagp = sqrt(fabs(fDesc))/(2*fA);
			printf("\nThe Roots are imaginary and they are\n");
			printf("\nRoot1 = %g+i%g\n",fRealp,fImagp);
			printf("\nRoot2 = %g-i%g\n",fRealp,fImagp);
	}
	return 0;
}

float fnCalcDiscriminant(float a, float b, float c)
{
	return (b*b-4*a*c);
}
\end{Verbatim}
\pagebreak
\section*{Output}
Run the following commands in your terminal:\\
\textbf{\$ gcc 01Quadratic.c -lm \\ \$./a.out }
\begin{Verbatim}
*************************************************************
*       PROGRAM TO FIND ROOTS OF A QUADRATIC EQUATION       *
*************************************************************
Enter the coefficients of a,b,c 
0 1 2

Invalid input, not a quadratic equation - try again
\end{Verbatim}
\textbf{\$./a.out }
\begin{Verbatim}
*************************************************************
*       PROGRAM TO FIND ROOTS OF A QUADRATIC EQUATION       *
*************************************************************
Enter the coefficients of a,b,c 
1 -5 6

The Roots are Real and distinct, they are 

Root1 = 3 and Root2 = 2
\end{Verbatim}
\textbf{\$./a.out }
\begin{Verbatim}
*************************************************************
*       PROGRAM TO FIND ROOTS OF A QUADRATIC EQUATION       *
*************************************************************
Enter the coefficients of a,b,c 
1 4 4

Roots are equal and the Roots are 

Root1 = -2 and Root2 = -2
\end{Verbatim}
\textbf{\$./a.out }
\begin{Verbatim}
*************************************************************
*       PROGRAM TO FIND ROOTS OF A QUADRATIC EQUATION       *
*************************************************************
Enter the coefficients of a,b,c 
1 3 3

The Roots are imaginary and they are

Root1 = -1.5+i0.866025
Root2 = -1.5-i0.866025
\end{Verbatim}

\chapter{}
\section{Determine Quadrant}
{\fontfamily{ppl}\selectfont \textbf{Write a program that takes the x and y coordinates of a point in the Cartesian plane and prints a message telling either the point is on the origin or an the axis on which the point lies or on the quadrant in which it is found.\\ }
Sample lines of output: \\
(-1.0, -2.5) is in quadrant III \\
(0.0, 4.8) is on the y-axis \\
}
\section*{C Code}
\begin{Verbatim}
#include <stdio.h>
#include <stdlib.h>

int main(void)
{
    float fXcor,fYcor;
    printf("\nEnter the coordinates\n");
    scanf("%g%g",&fXcor,&fYcor);

    if(0 == fXcor && 0 ==fYcor)
    {
        printf("\n(%g,%g) is at origin\n",fXcor,fYcor);
    }
    else if(0 == fXcor)
    {
        printf("\n(%g,%g) is on Y-axis\n",fXcor,fYcor);
    }
    else if(0 == fYcor)
    {
        printf("\n(%g,%g) is on X-axis\n",fXcor,fYcor);
    }
    else if(fXcor > 0 && fYcor > 0)
    {
        printf("\n(%g,%g) is in Quadrant I\n",fXcor,fYcor);
    }
    else if(fXcor < 0 && fYcor > 0)
    {
        printf("\n(%g,%g) is in Quadrant II\n",fXcor,fYcor);
    }
    else if(fXcor < 0 && fYcor < 0)
    {
        printf("\n(%g,%g) is in Quadrant III\n",fXcor,fYcor);
    }
    else
    {
        printf("\n(%g,%g) is in Quadrant IV\n",fXcor,fYcor);
    }

    return 0;
}
\end{Verbatim}
\section*{Output}
\begin{Verbatim}


Enter the coordinates
1.2 3.4

(1.2,3.4) is in Quadrant I


Enter the coordinates
2.3 -4.5

(2.3,-4.5) is in Quadrant IV


Enter the coordinates
-9.9 -6.5

(-9.9,-6.5) is in Quadrant III


Enter the coordinates
-6.5 7.5

(-6.5,7.5) is in Quadrant II


Enter the coordinates
5 0

(5,0) is on X-axis


Enter the coordinates
0 9.6

(0,9.6) is on Y-axis


Enter the coordinates
0 0

(0,0) is at origin


\end{Verbatim}
\pagebreak

\section{Sine Series}
{\fontfamily{ppl}\selectfont \textbf{ Write a program to find Sine of an angle using the series 
\[sin(x) = x - \frac{x^{3}}{3!} + \frac{x^{5}}{5!} - \frac{x^{7}}{7!}............\]
 For given N terms using the for loop structure. Also print sin(x) using Library Function.}}
\section*{C Code}
\begin{Verbatim}

#include <stdio.h>
#include <math.h>

#define PI 3.1416

int main(void)
{
	float fAngD, fAngR;
	float fTerm, fNum, fDen, fVal;
	int i,iNum;

	printf("Enter the Angle ....\n");
	scanf("%f",&fAngD);
	printf("Angle = %g\n",fAngD);

	printf("Enter the Number of terms...\n");
	scanf("%d",&iNum);
	printf("No of terms = %d\n",iNum);

	fAngR= (fAngD*PI)/180 ;

	fNum=fAngR;
	fDen=1.0;
	fVal =0.0;
	fTerm=fNum/fDen;
	for(i=1;i<=iNum;i++)
	{
		fVal = fVal + fTerm;
		fNum = -fNum * fAngR * fAngR ;
		fDen = fDen * (2*i) * (2*i+1);
		fTerm = fNum/fDen;
	}
	printf(" Calculated value is :sin(%g) = %g\n",fAngD,fVal);
	printf("Built In function value is :sin(%g) = %g\n",fAngD, sin(fAngR));
	return 0;
}

\end{Verbatim}
\pagebreak
\section*{Output}
\begin{Verbatim}

Enter the Angle ....
60
Angle = 60
Enter the Number of terms...
3
No of terms = 3
 Calculated value is :sin(60) = 0.866297
Built In function value is :sin(60) = 0.866027

Enter the Angle ....
60
Angle = 60
Enter the Number of terms...
8
No of terms = 8
 Calculated value is :sin(60) = 0.866027
Built In function value is :sin(60) = 0.866027

Enter the Angle ....
90
Angle = 90
Enter the Number of terms...
8
No of terms = 8
 Calculated value is :sin(90) = 1
Built In function value is :sin(90) = 1

\end{Verbatim}

\chapter{Print Calendar of a Month}
{\fontfamily{ppl}\selectfont \textbf{Write a program to display the calendar of a month whose starting day of the week and number of days in the month are given as input. (0,1,2,3...6 represent Sunday, Monday......Saturday respectively). Write a function \textit{printMonth} which takes these values as parameters and prints the calendar of the month. Perform input validation as well. 
}}
\section*{C Code}
\begin{Verbatim}
#include <stdio.h>
#include <stdlib.h>
void printMonth(int, int);

int main(void)
{
    int iNumOfDays, iStartDay;
    printf("\nEnter the day of the week on which the month starts\n");
    scanf("%d", &iStartDay);
    printf("\nEnter the number of days in the month\n");
    scanf("%d", &iNumOfDays);

    if(iStartDay < 0 || iStartDay > 6 || iNumOfDays >31 || iNumOfDays <28)
    {
        printf("\nInvalid Input\n");
        exit(0);
    }
    printMonth(iStartDay, iNumOfDays);
    return 0;
}
void printMonth(int iStDay, int iNoDays)
{
    int i;
    printf("\nSun\tMon\tTue\tWed\tThu\tFri\tSat\n");
    for(i=0;i<iStDay;i++)
        printf("\t");
    i=1;
    while(i<=iNoDays)
    {
        printf("%3d\t",i);
        if(0 == (i+iStDay)%7)
            printf("\n");
        i++;
    }
}

\end{Verbatim}
\section*{Output}
\begin{Verbatim}


Enter the day of the week on which the month starts
3

Enter the number of days in the month
31

Sun	Mon	Tue	Wed	Thu	Fri	Sat
                                   1	  2	  3	  4	
  5	  6	  7	  8	  9	 10	 11	
 12	 13	 14	 15	 16	 17	 18	
 19	 20	 21	 22	 23	 24	 25	
 26	 27	 28	 29	 30	 31	


Enter the day of the week on which the month starts
6

Enter the number of days in the month
30

Sun	Mon	Tue	Wed	Thu	Fri	Sat
						                    1	
  2	  3	  4	  5	  6	  7	  8	
  9	 10	 11	 12	 13	 14	 15	
 16	 17	 18	 19	 20	 21	 22	
 23	 24	 25	 26	 27	 28	 29	
 30	



Enter the day of the week on which the month starts
7

Enter the number of days in the month
28

Invalid Input
\end{Verbatim}


\chapter{Area under the curve y=log(x)}
{\fontfamily{ppl}\selectfont \textbf{Write a Program to calculate the area under the curve y = log(x) within a given left and right limit. Use trapezoidal approximation. Write an user defined function \textit{fnTrapArea} to calculate the area of a trapezoid. 
}}
\section*{C Code}
\begin{Verbatim}
#include <stdio.h>
#include <stdlib.h>
#include <math.h>

double fnAreaTrapezoid(double, double, double);

int main(void)
{
    double i, dDx=0.1,dX1,dX2,dY1,dY2;
    double dLeft, dRight, dArea=0.0, dTotArea=0.0;

    printf("\nEnter the left and right limit\n");
    scanf("%lf%lf", &dLeft, &dRight);

    printf("\nEnter the increment dx\n");
    scanf("%lf", &dDx);

    for(i=dLeft;i<dRight;i+=dDx)
    {
        dX1 = i;
        dX2 = i+dDx;
        dY1 = log(dX1);
        dY2 = log(dX2);

        dArea = fnAreaTrapezoid(dY1,dY2,dDx);
        dTotArea += dArea;
    }

    printf("\nArea under the curve is %lf units\n", dTotArea);
    return 0;
}

double fnAreaTrapezoid(double dHeight1, double dHeight2, double dBase)
{
    double dArea;
    dArea = dBase * ((dHeight1 + dHeight2)/2);
    return dArea;
}

\end{Verbatim}
\section*{Output}
\begin{Verbatim}


Enter the left and right limit
2 5

Enter the increment dx
0.001

Area under the curve is 3.660895 units



Enter the left and right limit
1 2

Enter the increment dx
0.001

Area under the curve is 0.386988 units



Enter the left and right limit
6 9

Enter the increment dx
0.01

Area under the curve is 6.046442 units

\end{Verbatim}

\chapter{Calculating nCr}
{\fontfamily{ppl}\selectfont \textbf{Write a recursive C function to find the factorial of a number, n!, defined by \textit{fact(n)=1, if n=0}. Otherwise \textit{fact(n)=n*fact(n-1)}. Using this function, Write a program to compute the binomial coefficient nCr. Perform input validation as well. 
}}
\section*{C Code}
\begin{Verbatim}
#include <stdio.h>
#include <stdlib.h>

int fnFactorial(int);

int main(void)
{
    int iN,iR,iNCR;
    printf("\nEnter the value of n\n");
    scanf("%d", &iN);
    printf("\nEnter the value of r\n");
    scanf("%d", &iR);

    if(iN < iR)
    {
        printf("\nnCr does not exist\n");
        exit(0);
    }

    iNCR = (fnFactorial(iN)/(fnFactorial(iR)*fnFactorial(iN-iR)));

    printf("\nFor n = %d and r = %d, %dC%d = %d\n", iN ,iR, iN, iR, iNCR);
    return 0;
}

int fnFactorial(int iVal)
{
    if(0 == iVal)
        return 1;
    else
        return (iVal * fnFactorial(iVal - 1));
}

\end{Verbatim}

\pagebreak
\section*{Output}
\begin{Verbatim}


Enter the value of n
3 6

Enter the value of r

nCr does not exist



Enter the value of n
5 2

Enter the value of r

For n = 5 and r = 2, 5C2 = 10



Enter the value of n
7 5

Enter the value of r

For n = 7 and r = 5, 7C5 = 21
\end{Verbatim}

\chapter{}
\section{Fibonacci Series}
{\fontfamily{ppl}\selectfont \textbf{Write a Program  to generate Fibonacci series of N numbers using recursion.
}}
\section*{C Code}
\begin{Verbatim}
#include <stdio.h>
#include <stdlib.h>

int fnFibonacci(int);

int main(void)
{
    int iTerm, i, iNum;

    printf("Enter the no of terms\n");
    scanf("%d", &iNum);

    printf("\nThe terms in the Fibonacci Sequence:\n");
    for(i=0;i<iNum;++i)
    {
        iTerm = fnFibonacci(i);
        printf("%d\t", iTerm);
    }
    return 0;
}

int fnFibonacci(int n)
{
    if(0 == n || 1 == n)
        return n;
    else
    {
        return fnFibonacci(n-1) + fnFibonacci(n-2);
    }
}

\end{Verbatim}
\section*{Output}
\begin{Verbatim}

Enter the no of terms
5

The terms in the Fibonacci Sequence:
0	1	1	2	3	

\end{Verbatim}

\section{x raised to the power n}
{\fontfamily{ppl}\selectfont \textbf{Write a Program in C to find the value of 'x' raised to the power 'n' using recursion.
}}

\section*{C Code}
\begin{Verbatim}
#include <stdio.h>
#include <stdlib.h>

int fnRecPow(int, int);
int main(void)
{
    int iX, iN, iRes;
    printf("\n Enter value of x and n\n");
    scanf("%d%d", &iX, &iN);

    //printf("\n%d raised to the power %d is %d\n",iX, iN, fnRecPow(iX,iN));
    iRes = fnRecPow(iX,iN);
        printf("\n%d raised to the power %d is %d\n",iX, iN, iRes);
    return 0;
}

int fnRecPow(int iX, int iN)
{
    if(1 == iN)
        return iX;
    else
        return (iX * fnRecPow(iX, iN-1));
}

\end{Verbatim}
\section*{Output}
\begin{Verbatim}
putta@putta-Lenovo-G580:~/Latex/Aug2nd/progs

 Enter value of x and n
4 5

4 raised to the power 5 is 1024
putta@putta-Lenovo-G580:~/Latex/Aug2nd/progs

 Enter value of x and n
9 9

9 raised to the power 9 is 387420489

\end{Verbatim}

\chapter{Bubble Sort and Binary Search}
{\fontfamily{ppl}\selectfont \textbf{Write a program that reads N integer numbers and arrange them in ascending order using Bubble Sort technique. Extend the program to perform a search operation on these sorted numbers by accepting a key element from the user applying Binary Search method. Report the result SUCCESS or FAILURE as the case may be. Write user defined functions for the following\\
\begin{enumerate}[i]
\item reads N integer numbers 
\item sorts N integer numbers in ascending order using  bubble sort 
\item display N integers 
\item search for a key element using Binary Search 
\end{enumerate}
}}
\section*{C Code}
\begin{Verbatim}
#include <stdio.h>
#include <stdlib.h>

int main(void)
{
    int iaArr[100],iKey,iNum,i,j,iLow,iMid,iHigh,iProbe,iTemp;
    printf("\nEnter the number of elements\n");
    scanf("%d",&iNum);

    printf("\nEnter %d elements\n", iNum);
    for(i=0;i<iNum;i++)
    {
        scanf("%d", &iaArr[i]);
    }

    for(i=0;i<iNum;i++)
    {
        for(j=i+1;j<iNum;j++)
        {
            if(iaArr[i] > iaArr[j])
            {
                iTemp = iaArr[i];
                iaArr[i] = iaArr[j];
                iaArr[j] = iTemp;
            }
        }
    }
    printf("\nSorted Input Array\n");
    for(i=0;i<iNum;i++)
    {
        printf("%d ", iaArr[i]);
    }

    printf("\nEnter the key element\n");
    scanf("%d", &iKey);

    iLow = 0;
    iHigh = iNum -1;
    iProbe = 0;
    while(iLow <= iHigh)
    {
        iProbe++;
        iMid = (iLow + iHigh)/2;
        if(iKey == iaArr[iMid])
        {
            printf("\nElement %d found at %d position\n",iKey, iMid+1);
            break;
        }
        else if(iKey < iaArr[iMid])
            iHigh = iMid-1;
        else
            iLow = iMid +1;
    }
    if(iLow > iHigh)
        printf("\nElement not found\n");

    printf("\nTotal number of probes required is %d\n", iProbe);

    return 0;
}

\end{Verbatim}


\section*{Output}
\begin{Verbatim}

Enter the number of elements
5        

Enter 5 elements
1 7 4 2 9

Sorted Input Array
1 2 4 7 9 
Enter the key element
4

Element 4 found at 3 position
Total number of probes required is 1


Enter the number of elements
3

Enter 3 elements
9 7 5

Sorted Input Array
5 7 9 
Enter the key element
8

Element not found
Total number of probes required is 2
\end{Verbatim}

\chapter{Selection Sort and Insertion Sort}
{\fontfamily{ppl}\selectfont \textbf{Write a program to implement sorting with the following user defined functions. 
\begin{enumerate}[i]
\item function to generate n random numbers in the range 0 to 999 and store it in an array. 
\item function to sort the array using Selection sort. 
\item function to sort the array using Insertion sort. 
\item function to display the elements of an array. 
\end{enumerate}
}}
\section*{C Code}
\begin{Verbatim}
/********************************************************************************
*File		: InsertSelectSort.c
*Description	: Program to sort an array using Insertion Sort
*Author		: Prabodh C P
*Compiler	: gcc compiler 4.6.3, Ubuntu 12.04
*Date		: 22 Nov 2013
********************************************************************************/

#include <stdio.h>
#include <stdlib.h>
#include <time.h>

void fnGenRandInput(int [], int);
void fnDispArray( int [], int);
void fnInsertionSort(int [], int);
void fnSelectSort(int [], int);


/******************************************************************************
*Function	: main
*Input parameters: no parameters
*RETURNS	:	0 on success
******************************************************************************/
int main(void)
{

	int iaArr[100],iNum;

	printf("\nInsertion Sort\n");
	printf("\nEnter the number of elements to sort\n");
	scanf("%d",&iNum);
	fnGenRandInput(iaArr,iNum);
	printf("\nUnsorted Array\n");
	fnDispArray(iaArr,iNum);
	fnInsertionSort(iaArr,iNum);
	printf("\nSorted Array\n");
	fnDispArray(iaArr,iNum);

	printf("\nSelection Sort\n");
	printf("\nEnter the number of elements to sort\n");
	scanf("%d",&iNum);
	fnGenRandInput(iaArr,iNum);
	printf("\nUnsorted Array\n");
	fnDispArray(iaArr,iNum);
	fnSelectSort(iaArr,iNum);
	printf("\nSorted Array\n");
	fnDispArray(iaArr,iNum);

	return 0;
}

/******************************************************************************
*Function	: fnSelectSort
*Description	: Function to sort elements in an iaArray using Quick Sort
*Input parameters:
*	int A[] - iaArray to hold integers
*	int n	- no of elements in the array
*RETURNS	: no value
******************************************************************************/

void fnSelectSort(int A[], int n)
{
	int i,j,iMinPos,iTemp;
	for(i=0;i<n;i++)
	{
		iMinPos = i;
		for(j=i+1;j<n;j++)
		{
			if(A[j]<A[iMinPos])
				iMinPos = j;
		}
		iTemp = A[i];
		A[i] = A[iMinPos];
		A[iMinPos] = iTemp;
	}
}

/******************************************************************************
*Function	: fnInsertionSort
*Description	: Function to sort an array using Insertion Sort
*Input parameters:
*	int A[] - iaArray to hold integers
*	int n	- no of elements in the array
*RETURNS	: no value
******************************************************************************/

void fnInsertionSort(int A[], int n)
{
	int i, j, iKey;
	for(i=1;i<n;i++)
	{
		iKey = A[i];
		j = i-1;
		while(j>=0 && A[j] > iKey)
		{
			A[j+1] = A[j];
			j--;
		}
		A[j+1] = iKey;
	}
}

/******************************************************************************
*Function	: GenRandInput
*Description	: Function to generate a fixed number of random elements
*Input parameters:
*	int X[] - array to hold integers
*	int n	- no of elements in the array
*RETURNS	:no return value
******************************************************************************/

void fnGenRandInput(int X[], int n)
{
	int i;
	srand(time(NULL));
	for(i=0;i<n;i++)
	{
		X[i] = rand()%1000;
	}
}

/******************************************************************************
*Function	: DispArray
*Description	: Function to display elements of an array
*Input parameters:
*	int X[] - array to hold integers
*	int n	- no of elements in the array
*RETURNS	: no return value
******************************************************************************/

void fnDispArray( int X[], int n)
{
	int i;
	for(i=0;i<n;i++)
		printf(" %4d \n",X[i]);
}

\end{Verbatim}

\pagebreak
\section*{Output}
\begin{Verbatim}


Insertion Sort

Enter the number of elements to sort
5

Unsorted Array
  845 
  154 
  879 
  668 
  214 

Sorted Array
  154 
  214 
  668 
  845 
  879 

Selection Sort

Enter the number of elements to sort
5

Unsorted Array
  925 
  281 
  200 
  162 
  484 

Sorted Array
  162 
  200 
  281 
  484 
  925 

\end{Verbatim}


\part{PART B}

\chapter{}
\section{Set Difference}
{\fontfamily{ppl}\selectfont \textbf{Write a program to find the Difference of two sets(sets have distinct elements)    }}
\section*{C Code}
\begin{Verbatim}

#include <stdio.h>
#include <stdlib.h>

#define TRUE 1
#define FALSE 0
int main(void)
{
    int iaA[10], iaB[10], iaC[10], iNum1, iNum2;
    int i, j, iNotOccur, iCount;

    printf("\nEnter no of elements in Array A\n");
    scanf("%d",&iNum1);
    printf("\nEnter no of elements in Array B\n");
    scanf("%d",&iNum2);

    printf("\nEnter %d elements in Array A\n",iNum1);
    for(i=0;i<iNum1;++i)
    {
	scanf("%d",&iaA[i]);
    }
    printf("\nEnter %d elements in Array B\n",iNum2);
    for(i=0;i<iNum2;++i)
    {
	scanf("%d",&iaB[i]);
    }

	/*Copy those elements of X that are not in Y into C*/
    for(i=0;i<iNum1;i++)
    {
	iNotOccur = TRUE;
	for(j=0;j<iNum2;j++)
	{
	    if(iaA[i] == iaB[j])
	        iNotOccur = FALSE;
	}
	if(iNotOccur == TRUE)
	{
	    iaC[iCount++] = iaA[i];
	}
    }
    printf("\nElements of Array A:\n");
    for(i=0;i<iNum1;++i)
    {
	printf("%d\t",iaA[i]);
    }

    printf("\nElements of Array B:\n");
    for(i=0;i<iNum2;++i)
    {
	printf("%d\t",iaB[i]);
    }

    printf("\nDifference of A and B: \n");
    for(i=0;i<iCount;++i)
    {
	printf("%d\t",iaC[i]);
    }
    printf("\n");
    return 0;
}
\end{Verbatim}
\section*{Output}
\begin{Verbatim}

Enter no of elements in Array A
4     

Enter no of elements in Array B
5

Enter 4 elements in Array A
1 2 3 4

Enter 5 elements in Array B
3 4 5 6 7  

Elements of Array A:
1	2	3	4	
Elements of Array B:
3	4	5	6	7	
Difference of A and B: 
1	2	


Enter no of elements in Array A
5 

Enter no of elements in Array B
4 

Enter 5 elements in Array A
3 4 5 6 7

Enter 4 elements in Array B
1 2 3  4

Elements of Array A:
3	4	5	6	7	
Elements of Array B:
1	2	3	4	
Difference of A and B: 
5	6	7	

Enter no of elements in Array A
5 

Enter no of elements in Array B
4

Enter 5 elements in Array A
3 5 7 4 6

Enter 4 elements in Array B
4 1 2 3

Elements of Array A:
3	5	7	4	6	
Elements of Array B:
4	1	2	3	
Difference of A and B: 
5	7	6	

\end{Verbatim}
\pagebreak
\section{Duplicate element check in an array of elements}
{\fontfamily{ppl}\selectfont \textbf{Write a program to find whether an array contains duplicate elements or not.}}
\section*{C Code}
\begin{Verbatim}
#include <stdio.h>
#include <stdlib.h>

int main(void)
{
    int iaA[10], iNum;
    int i, j;

    printf("\nEnter no of elements in Array \n");
    scanf("%d",&iNum);

    printf("\nEnter %d elements in Array A\n",iNum);
    for(i=0;i<iNum;++i)
    {
        scanf("%d",&iaA[i]);
    }

    for(i=0;i<iNum;++i)
    {
        for(j=i+1;j<iNum;++j)
        {
            if(iaA[i] == iaA[j])
            {
                printf("\nArray contains duplicate elements\n");
                exit(0);
            }
        }

    }
    printf("\nArray does not contain duplicate elements\n");
    return 0;
}

\end{Verbatim}
\section*{Output}
\begin{Verbatim}
putta@putta-Lenovo-G580:~/Latex/Aug2nd/progs

Enter no of elements in Array 
5

Enter 5 elements in Array A
1 2 3 1 2

Array contains duplicate elements
putta@putta-Lenovo-G580:~/Latex/Aug2nd/progs

Enter no of elements in Array 
5 

Enter 5 elements in Array A
2 4 6 8 9

Array does not contain duplicate elements

\end{Verbatim}

\chapter{Matrix Construction}
{\fontfamily{ppl}\selectfont \textbf{Write a program that reads a matrix of order M X N and fills the matrix by following: 
\begin{enumerate}[i]
\item Upper left triangle with '1' s 
\item Lower right triangle with  '-1' s 
\item Right to left diagonal with zeros \\
\end{enumerate}
Check the necessity condition needed and display the contents of the matrix in proper format. 
}}
\section*{C Code}
\begin{Verbatim}
#include<stdio.h>
#include<stdlib.h>

int main(void)
{
	int iaMat[100][100],iRow,iCol,i,j;

	printf("\nEnter the order of the matrix\n");
	scanf("%d%d",&iRow,&iCol);

	if(iRow != iCol)
	{
		printf("\nMatrix should be a Square Matrix\n");
		exit(0);
	}
	for(i=0;i<iRow;i++)
	{
		for(j=0;j<iCol;j++)
		{
			if(i < j)
				iaMat[i][j] = 1;
			else if(i == j)
				iaMat[i][j] = 0;
			else
				iaMat[i][j] = -1;
		}
	}
	printf("\nMatrix \n");
	for(i=0;i<iRow;i++)
	{
		for(j=0;j<iCol;j++)
		{
			printf("%2d\t",iaMat[i][j]);
		}
		printf("\n");
	}
	return 0;
}
\end{Verbatim}
\section*{Output}
\begin{Verbatim}


Enter the order of the matrix
4 5

Matrix should be a Square Matrix



Enter the order of the matrix
5 5

Matrix 
 0	 1	 1	 1	 1	
-1	 0	 1	 1	 1	
-1	-1	 0	 1	 1	
-1	-1	-1	 0	 1	
-1	-1	-1	-1	 0	
\end{Verbatim}

\chapter{Matrix Multiplication}
{\fontfamily{ppl}\selectfont \textbf{Write a program that reads two matrices A (m x n ) and B (p x q ) and Compute the product A X  B. Read matrix A in row major order and matrix B in column major order. Print both the input matrices and resultant matrix with suitable headings and in matrix format. Display the trace of the matrix if it exists. Program must check the compatibility of orders of the matrices for multiplication. Report appropriate message in case of incompatibility. 
}}
\section*{C Code}
\begin{Verbatim}
/***************************************************************************
*File		: MatrixMul.c
*Description	: Program to implement Matrix Multiplication
*Author		: Prabodh C P
*Compiler	: gcc 4.6.3 compiler, Ubuntu 10.04
*Date		: Wednesday, July 16 2014
***************************************************************************/

#include<stdio.h>
#include<stdlib.h>

/***************************************************************************
*Function	: 	main
*Input parameters:	no parameters
*RETURNS	:	0 on success
***************************************************************************/

int main(void)
{
	int iM, iN, iP, iQ, i, j, k, iaMat1[10][10], iaMat2[10][10];
	int iaProd[10][10] = {0};   //Initialize the matrix with all zeros


	printf("\n*********************************************************");
	printf("\n*\tPROGRAM TO IMPLEMENT MATRIX MULIPLICATION\t*\n");
	printf("*********************************************************");

	printf("\nEnter the order of Matrix1\n");
	scanf("%d%d",&iM,&iN);

	printf("\nEnter the order of Matrix2\n");
	scanf("%d%d",&iP,&iQ);

	if( iN != iP)
	{
		printf("\nMatrix Multiplication not possible\n");
		exit(0);
	}


	printf("\nEnter the elements of Matrix 1 in row major order\n");
	for(i=0;i<iM;i++)
		for(j=0;j<iN;j++)
			scanf("%d",&iaMat1[i][j]);

	printf("\nEnter the elements of Matrix 2 in column major order\n");
	for(i=0;i<iQ;i++)
		for(j=0;j<iP;j++)
			scanf("%d",&iaMat2[j][i]);


	for(i=0;i<iM;i++)
	{
		for(j=0;j<iQ;j++)
		{
			for(k=0;k<iN;k++)
			{
			iaProd[i][j] += iaMat1[i][k] * iaMat2[k][j];
			}
		}
	}

/**************************************************************************************************
                |*|                    |*|
a00  a01  a02   |*|    b00  b01  b02   |*|
                |*|                    |*|
a10  a11  a12   |*|    b10  b11  b12   |*|
                |*|                    |*|
a20  a21  a22   |*|    b20  b21  b22   |*|
                |*|                    |*|

(a00*b00+a01*b10+a02*b20) (a00*b01+a01*b11+a02*b21) (a00*b02+a01*b12+a02*b22)
(a10*b00+a11*b10+a12*b20) (a10*b01+a11*b11+a12*b21) (a10*b02+a11*b12+a12*b22)
(a20*b00+a21*b10+a22*b20) (a20*b01+a21*b11+a22*b21) (a20*b02+a21*b12+a22*b22)
**************************************************************************************************/


	printf("\nMatrix 1\n");

	for(i=0;i<iM;i++)
	{
		for(j=0;j<iN;j++)
		{
			printf("%d\t",iaMat1[i][j]);
		}
		printf("\n");
	}
	printf("\n");

	printf("\nMatrix 2\n");

	for(i=0;i<iP;i++)
	{
		for(j=0;j<iQ;j++)
		{
			printf("%d\t",iaMat2[i][j]);
		}
		printf("\n");
	}
	printf("\n");

	printf("\nThe Product matrix is is \n");

	for(i=0;i<iM;i++)
	{
		for(j=0;j<iQ;j++)
		{
			printf("%d\t",iaProd[i][j]);
		}
		printf("\n");
	}
	printf("\n");
	return 0;
}

\end{Verbatim}


\section*{Output}
\begin{Verbatim}

*********************************************************
*	PROGRAM TO IMPLEMENT MATRIX MULIPLICATION	*
*********************************************************
Enter the order of Matrix1
2 3

Enter the order of Matrix2
3 2

Enter the elements of Matrix 1 in row major order
1 2 3 4 5 6

Enter the elements of Matrix 2 in column major order
1 2 3 4 5 6

Matrix 1
1	2	3	
4	5	6	


Matrix 2
1	4	
2	5	
3	6	


The Product matrix is is 
14	32	
32	77	


*********************************************************
*	PROGRAM TO IMPLEMENT MATRIX MULIPLICATION	*
*********************************************************
Enter the order of Matrix1
3 4

Enter the order of Matrix2
3 4

Matrix Multiplication not possible

\end{Verbatim}

\chapter{Swap smallest and largest elements in an array using pointers}
{\fontfamily{ppl}\selectfont \textbf{Write a program to find the smallest and largest elements in an array using pointers and then swap these elements and display the resultant array. 
}}
\section*{C Code}
\begin{Verbatim}
#include <stdio.h>
#include <stdlib.h>

int main(void)
{
    int iaList[50], iNum, i, j,iSmall, iLarge,iTemp, iPos;
    int *iPtr1,*iPtr2;

    iPtr1 = iPtr2 = iaList;
    printf("\nEnter the value of n : ");
    scanf("%d", &iNum);

    printf("\nEnter the elements: ");
    for(i=0;i<iNum;++i)
    {
        scanf("%d", &iaList[i]);
    }

    iSmall = *iPtr1;
    iLarge = *iPtr2;
    for(i=1;i<iNum;++i)
    {
        if(iaList[i] < iSmall)
        {
            iSmall = iaList[i];
            iPtr1 = &iaList[i];
        }
        if(iaList[i] > iLarge)
        {
            iLarge = iaList[i];
            iPtr2 = &iaList[i];
        }
    }

    printf("\nArray elements before swapping the largest and smallest elements\n");
    for(i=0;i<iNum;i++)
    {
        printf("%d\t", iaList[i]);
    }

    printf("\nSmallest Element = %d\n",*iPtr1);
    printf("\nLargest Element = %d\n",*iPtr2);

    iTemp = *iPtr1;
    *iPtr1 = *iPtr2;
    *iPtr2 = iTemp;

    printf("\nArray elements after swapping the largest and smallest elements\n");
    for(i=0;i<iNum;i++)
    {
        printf("%d\t", iaList[i]);
    }
    return 0;
}

\end{Verbatim}
\section*{Output}
\begin{Verbatim}


Enter the value of n : 5

Enter the elements: 5 4 3 2 1

Array elements before swapping the largest and smallest elements
5	4	3	2	1	
Smallest Element = 1

Largest Element = 5

Array elements after swapping the largest and smallest elements
1	4	3	2	5	



Enter the value of n : 6

Enter the elements: 3 2 1 4 5 6

Array elements before swapping the largest and smallest elements
3	2	1	4	5	6	
Smallest Element = 1

Largest Element = 6

Array elements after swapping the largest and smallest elements
3	2	6	4	5	1	

\end{Verbatim}

\chapter{Pass by Reference}
{\fontfamily{ppl}\selectfont \textbf{Write a program that reads the length, breadth and height of a cuboid. The program has to calculate the surface area and volume of the cuboid using the function \textit{fnCalcVolSurfArea}. The results are then to be displayed by the main function. (Hint: use pass by reference) 
}}
\section*{C Code}
\begin{Verbatim}
#include <stdio.h>
#include <stdlib.h>

void fnCalcVolSurfArea(int, int, int, int*, int*);

int main(void)
{
    int iLength, iBreadth, iHeight, iSurfArea, iVolume;
    printf("\nEnter the length, breadth and height of the cuboid\n");
    scanf("%d%d%d", &iLength, &iBreadth, &iHeight);
    fnCalcVolSurfArea(iLength, iBreadth, iHeight, &iSurfArea, &iVolume);
    printf("\nSurface Area of cuboid is %d units\n", iSurfArea);
    printf("\nVolume of cuboid is %d cubic units\n", iVolume);
    return 0;
}
void fnCalcVolSurfArea(int iL, int iB, int iH, int *iA, int *iV)
{
    *iA = 2*iL*iB + 2*iB*iH + 2*iH*iL;
    *iV = iL*iB*iH;
}
\end{Verbatim}
\section*{Output}
\begin{Verbatim}


Enter the length, breadth and height of the cuboid
3 4 5

Surface Area of cuboid is 94 units

Volume of cuboid is 60 cubic units


Enter the length, breadth and height of the cuboid
6 6 6

Surface Area of cuboid is 216 units

Volume of cuboid is 216 cubic units
\end{Verbatim}

\chapter{Caesar Cipher}
{\fontfamily{ppl}\selectfont \textbf{Write a Program to implement Caesar Cipher.  Input a message and then encodes it by replacing each character in the message by a character that is three positions ahead in the English alphabet sequence, wrap back to 'a' if the character is 'z'. Display the encoded message. Decode the message using the inverse procedure and display it. 
}}
\section*{C Code}
\begin{Verbatim}
#include <stdio.h>
#include <stdlib.h>
#include <ctype.h>
void fnEncrypt(char []);
void fnDecrypt(char []);

int main(void)
{
    char acMesg[50];

/*    char ch = 'a';
    int i;
    printf("\nCharacter\tEncrypt\t\tDecrypt\n");
    for(i=0;i<26;i++)
    {
        printf("%c=%d\t\t%4c\t\t%4c\n", ch, ch, ((ch-'a'+3+26)%26+'a'), ((ch-'a'-3+26)%26+'a'));
        ch=ch+1;
    }
    getchar();
*/
    printf("\nEnter your plain text in lower case\n");
    gets(acMesg);

    fnEncrypt(acMesg);
    printf("Encrypted Ciphertext\n");
    puts(acMesg);

    fnDecrypt(acMesg);
    printf("Decrypted Plaintext\n");
    puts(acMesg);

    return 0;
}

void fnEncrypt(char acStr[50])
{
    int i;
    for(i=0;acStr[i]!='\0';i++)
    {
        if(!isspace(acStr[i]))
        acStr[i] = (acStr[i]-'a'+3+26)%26+'a';
    }
}
void fnDecrypt(char acStr[50])
{
    int i;
    for(i=0;acStr[i]!='\0';i++)
    {
        if(!isspace(acStr[i]))
        acStr[i] = (acStr[i]-'a'-3+26)%26+'a';
    }
}

\end{Verbatim}
\section*{Output}
\begin{Verbatim}


Enter your plain text in lower case
ramanu kaadige hodanu
Encrypted Ciphertext
udpdqx nddgljh krgdqx
Decrypted Plaintext
ramanu kaadige hodanu


Enter your plain text in lower case
you are what you think you are    
Encrypted Ciphertext
brx duh zkdw brx wklqn brx duh
Decrypted Plaintext
you are what you think you are

\end{Verbatim}

\chapter{Substring Search}
{\fontfamily{ppl}\selectfont \textbf{Write a program to read a line of text from the keyboard and print the number of occurrences of a given substring using the builtin function strstr(). 
}}
\section*{C Code}
\begin{Verbatim}
#include <stdio.h>
#include <stdlib.h>
#include <string.h>

int main(void)
{
    char acText[100], acPattern[10];
    int iCount;
    char *pos = NULL, *start = NULL;
    printf("\nEnter the text\n");
    gets(acText);

    printf("\nEnter the substring to search for\n");
    gets(acPattern);

    start = acText;
    while((pos = strstr(start, acPattern)))
    {

        iCount++;
        printf("\nSubstring found at position %ld\n", pos-acText+1);
        start = pos+1;
    }

    printf("\nNo of occurences is %d\n", iCount);

    return 0;
}

\end{Verbatim}

\pagebreak
\section*{Output}
\begin{Verbatim}


Enter the text
The cat with five legs ran down the hall    

Enter the substring to search for
he

Substring found at position 2

Substring found at position 34

No of occurences is 2



Enter the text
Freedom of the mind

Enter the substring to search for
faith

No of occurences is 0

\end{Verbatim}

\chapter{String Comparison}
{\fontfamily{ppl}\selectfont \textbf{Write a program to read two strings and then determine whether they are identical or not using a user defined function fnMyStrCmp which takes these strings as parameters. Display the result in the main function 
}}
\section*{C Code}
\begin{Verbatim}
#include <stdio.h>
#include <stdlib.h>
#include <stdbool.h>

bool fnMyStrCmp(char [], char []);

int main(void)
{
    char acStr1[50],acStr2[50];
    bool notIdentical;

    printf("\nEnter string1\n");
    gets(acStr1);

    printf("\nEnter string2\n");
    gets(acStr2);

    notIdentical = fnMyStrCmp(acStr1, acStr2);

    if(notIdentical)
        printf("\nThe strings are not identical\n");
    else
        printf("\nThe strings are identical\n");

    return 0;
}

bool fnMyStrCmp(char str1[], char str2[])
{
    int i,j;
    bool notSame = false;
    for(i=0; str1[i] != '\0';i++);
    for(j=0; str2[j] != '\0';j++);

    if(i != j)
    {
        notSame = true;
        return notSame;
    }
    for(i=0; str1[i] != '\0';i++)
    {
        if(str1[i] != str2[i])
        {
            notSame = true;
            return notSame;
        }
    }
    return notSame;
}

\end{Verbatim}
\section*{Output}
\begin{Verbatim}


Enter string1
Tumkur

Enter string2
Bangalore

The strings are not identical



Enter string1
Tumkur

Enter string2
TUMKUR

The strings are not identical



Enter string1
tumkur

Enter string2
tumkur

The strings are identical



Enter string1
Tumkur

Enter string2
Tumkur City

The strings are not identical

\end{Verbatim}

\chapter{Bowler Information}
{\fontfamily{ppl}\selectfont \textbf{Write a program that stores the details of 5 bowlers given by the user. The following are the information stored for each bowler
\begin{enumerate}[a]
\item Name	
\item Nationality	
\item No of matches played	
\item No of Wickets taken 
\item No of overs bowled 	
\item No of runs conceded 
\end{enumerate}
Then find and display details of 
\begin{enumerate}[i]
\item the most economic bowler 
\item the bowler with the best strike rate 
\end{enumerate}
 }}
\section*{C Code}
\begin{Verbatim}
#include <stdio.h>
#include <stdlib.h>

typedef struct
{
    char acName[30];
    char acNationality[30];
    int iMatches, iWickets, iOvers, iRuns;
}BOWLER;

int main(void)
{
    BOWLER bowlers[5];
    int i, iPosEconomy=0, iPosStrRate=0;
    float fEconomy[5], fStrRate[5],fBestEconomy,fBestStrRate;
    printf("\nEnter details\n");

    for(i=0;i<5;i++)
    {
        printf("Name : ");
        scanf("%s", bowlers[i].acName);
        printf("Nationality : ");
        scanf("%s",bowlers[i].acNationality);
        printf("Matches Played : ");
        scanf("%d", &bowlers[i].iMatches);
        printf("Wickets taken : ");
        scanf("%d", &bowlers[i].iWickets);
        printf("Overs Bowled : ");
        scanf("%d", &bowlers[i].iOvers);
        printf("Runs Conceded : ");
        scanf("%d", &bowlers[i].iRuns);
    }

    fBestEconomy = fEconomy[0] = (float)bowlers[0].iRuns/bowlers[0].iOvers;
    fBestStrRate = fStrRate[0] = (float)bowlers[0].iOvers/bowlers[0].iWickets;
    printf("\n\nName\t\tNation\t\tMatches  Wickets   Overs  Runs  Economy   SR\n");

    for(i=0;i<5;i++)
    {
        fEconomy[i] = (float)bowlers[i].iRuns/bowlers[i].iOvers;
        fStrRate[i] = (float)bowlers[i].iOvers/bowlers[i].iWickets;
        printf("%-16s", bowlers[i].acName);
        printf("%-10s\t",bowlers[i].acNationality);
        printf("%3d", bowlers[i].iMatches);
        printf("%9d", bowlers[i].iWickets);
        printf("%12d", bowlers[i].iOvers);
        printf("%6d", bowlers[i].iRuns);
        printf("%8.2f",fEconomy[i]);
        printf("%8.2f",fStrRate[i]);
        printf("\n");

        if(fBestEconomy > fEconomy[i])
            iPosEconomy = i;
        if(fBestStrRate > fStrRate[i])
            iPosStrRate = i;
    }
    printf("\nThe Bowler with the best economy is : %s\n", bowlers[iPosEconomy].acName);
    printf("\nThe Bowler with the best Strike rate is : %s\n", bowlers[iPosStrRate].acName);
    return 0;
}

\end{Verbatim}

\pagebreak
\section*{Output}
\begin{Verbatim}
Enter details

Name : Ramesh      
Nationality : India
Matches Played : 123
Wickets taken : 99
Overs Bowled : 1200
Runs Conceded : 4567

Name : Mushtaq
Nationality : Bangladesh
Matches Played : 56
Wickets taken : 50
Overs Bowled : 510
Runs Conceded : 2200

Name : Prakash
Nationality : India
Matches Played : 212
Wickets taken : 176
Overs Bowled : 2096
Runs Conceded : 6036

Name : Stewart
Nationality : England
Matches Played : 98
Wickets taken : 65
Overs Bowled : 910
Runs Conceded : 7122

Name : Chandana
Nationality : Srilanka
Matches Played : 154
Wickets taken : 136
Overs Bowled : 1321
Runs Conceded : 9999

Name	    Nation	Matches  Wickets       Overs  Runs  Economy   SR
Ramesh          India     	123       99        1200  4567    3.81   12.12
Mushtaq         Bangladesh	 56       50         510  2200    4.31   10.20
Prakash         India     	212      176        2096  6036    2.88   11.91
Stewart         England   	 98       65         910  7122    7.83   14.00
Chandana        Srilanka  	154      136        1321  9999    7.57    9.71

The Bowler with the best economy is : Prakash

The Bowler with the best Strike rate is : Chandana

\end{Verbatim}

\chapter{Distance between two points}
{\fontfamily{ppl}\selectfont \textbf{Write a program that create a structure to represent a point in a cartesian plane. The program has to read coordinates of two points as input. Using a user defined function \textit{fnCalcDistance} which takes these points as parameters calculate the distance between the two points. Display the result in the main function. (Hint : use Distance formula)
}}
\section*{C Code}
\begin{Verbatim}
#include <stdio.h>
#include <stdlib.h>
#include <math.h>

typedef struct
{
    int iXcor, iYcor;
}POINT;


int main(void)
{
	POINT p1, p2;

	float fDistance;
	
	int iDiffX,iDiffY;

	printf("\nEnter coordinates of the first point\n");
	scanf("%d%d",&p1.iXcor,&p1.iYcor);

	printf("\nEnter coordinates of the second point\n");
	scanf("%d%d",&p2.iXcor,&p2.iYcor);

	iDiffX = p2.iXcor - p1.iXcor;
	iDiffY = p2.iYcor - p1.iYcor;

	fDistance = sqrt(((iDiffX)*(iDiffX)) + ((iDiffY)*(iDiffY)));

	printf("\nDistance between the two points is %g units\n",fDistance);

	return 0;
}

\end{Verbatim}

\pagebreak

\section*{Output}
\begin{Verbatim}


Enter coordinates of the first point
0 0

Enter coordinates of the second point
5 0

Distance between the two points is 5 units


Enter coordinates of the first point
5 5

Enter coordinates of the second point
5 9

Distance between the two points is 4 units


Enter coordinates of the first point
1 1

Enter coordinates of the second point
6 7

Distance between the two points is 7.81025 units

\end{Verbatim}

\chapter{Bitwise Operations}
{\fontfamily{ppl}\selectfont \textbf{Write a program that reads an unsigned integer and then perform the following operations using user defined functions 
\begin{enumerate}[i]
\item \textit{fnSetBit} - sets a specified bit to 1
\item \textit{fnMaskBit} - masks a specified bit to 0  
\end{enumerate}
}}
\section*{C Code}
\begin{Verbatim}
#include <stdio.h>
#include <stdlib.h>

unsigned short fnSetBit(unsigned short, unsigned short);
unsigned short fnMaskBit(unsigned short, unsigned short);

int main(void)
{
	unsigned short uNum,uPos;

	printf("\nEnter a value\n");
	scanf("%hu",&uNum);

	printf("\nEnter bit uPosition (1-16) to set to one\n");
	scanf("%hu",&uPos);

	uNum = fnSetBit(uNum, uPos);

	printf("\nValue after setting the specified bit = %hu\n",uNum);

	printf("\nEnter a value\n");
	scanf("%hu",&uNum);

	printf("\nEnter bit uPosition (1-16) to mask to zero\n");
	scanf("%hu",&uPos);

	uNum = fnMaskBit(uNum, uPos);

	printf("\nValue after masking the specified bit = %hu\n",uNum);

	return 0;
}

unsigned short fnSetBit(unsigned short uVal, unsigned short uPos)
{
	unsigned short uMask=1;
	uMask = uMask << (uPos-1);
	uVal = uVal | uMask;
	return uVal;
}
unsigned short fnMaskBit(unsigned short uVal, unsigned short uPos)
{
	unsigned short uMask=1;
	uMask = uMask << (uPos-1);
	uMask = ~uMask;
	uVal = uVal & uMask;
	return uVal;
}
\end{Verbatim}
\section*{Output}
\begin{Verbatim}


Enter a value
56

Enter bit uPosition (1-16) to set to one
3

Value after setting the specified bit = 60

Enter a value
63

Enter bit uPosition (1-16) to mask to zero
3

Value after masking the specified bit = 59

\end{Verbatim}

\end{document}
