\documentclass[a4paper]{report}
\usepackage{fancyvrb}
\usepackage[margin=2cm]{geometry}
\usepackage{url}
\begin{document}
\title{Computer Programming Laboratory Manual}
\author{\textbf{Prabodh C P}\\\textit{Volunteer}\\Free Software Movement Karnataka\\ \url{www.fsmk.org}}
\maketitle
\tableofcontents

\chapter{Roots of a Quadratic Equation Program}
{\fontfamily{pbk}\selectfont \textbf{Design and develop a flowchart or an algorithm that takes three coefficients (a, b, and c) of a Quadratic equation \textbf{\(ax^{2}+bx+c=0\)} as input and compute all possible roots. Implement a C program for the developed flowchart/algorithm and execute the same to output the possible roots for a given set of coefficients with appropriate messages.
}}


\section*{Summary}
Any quadratic equation has two roots and the roots of the equation can be found using the formula
\[x=\frac{-b\pm\sqrt{discriminant}}{2*a}\] where \(discriminant={b^{2}-4*a*c}\) and a,b and c are the coefficients of the quadratic equation \textbf{\(ax^{2}+bx+c=0\)}\\
Hence as per the equation, if discriminant is equal to 0, then the value of both the roots are equal and real. Also to find x, when discriminant is not 0, we need to find square root of disc. When discriminant value is less than 0, the square root of the discriminant is imaginary and hence the roots of the equation are supposed to be imaginary and distinct. When discriminant value is greater than 0, the square root of the discriminant is real and hence the roots of the equation are supposed to be real and distinct.

\section*{Algorithm}
\begin{Verbatim}
Quadratic_Equation(a,b,c)
//Finds the roots of a Quadratic Equation
//Input: Coefficients a, b & c
//Output: x1 and x2 the roots of the Quadratic Equation
if a == 0
	write "Not a Quadratic Equation"
	x = -b/c
	return x

desc = (b*b)-(4*a*c)

if desc == 0
	write "Roots are real and equal"
	x1 = x2 = -b/(2*a)
else if desc > 0
	write "Roots are real and distinct"
	x1 = (-b + sqrt(desc))/(2*a)
	x2 = (-b – sqrt(desc))/(2*a)
else
	write "Roots are imaginary"
	realp = -b/(2*a)
	imagp = sqrt(-desc)/(2*a)
	x1 = realp + i imagp
	x2 = realp - i imagp

return x1 and x2

\end{Verbatim}

\section*{C Code}
\begin{Verbatim}
/***************************************************************************
*File : 01Quadratic.c
*Description : Program to find the roots of a Quadratic Equation
*Author : Prabodh C P
*Compiler : gcc compiler 4.6.3, Ubuntu 12.04
*Date : 16 July 2014
***************************************************************************/

#include<stdio.h>
#include<stdlib.h>
#include<math.h>

/***************************************************************************
*Function : main
*Input parameters : no parameters
*RETURNS : 0 on success
***************************************************************************/

int main(void)
{
	float fA,fB,fC,fDesc,fX1,fX2,fRealp,fImagp;

	printf("\n*************************************************************");
	printf("\n*\tPROGRAM TO FIND ROOTS OF A QUADRATIC EQUATION\t    *\n");
	printf("*************************************************************");


	printf("\nEnter the coefficients of a,b,c \n");
	scanf("%f%f%f",&fA,&fB,&fC);
	if(0 == fA)
	{
		printf("\nInvalid input, not a quadratic equation - try again\n");
		exit(0);
	}

	/*COMPUTE THE DESCRIMINANT*/
	fDesc=fB*fB-4*fA*fC;

	if(0 == fDesc)
	{
		fX1 = fX2 = -fB/(2*fA);

		printf("\nRoots are equal and the Roots are \n");
		printf("\nRoot1 = %g and Root2 = %g\n",fX1,fX2);
	}
	else if(fDesc > 0)
	{
		fX1 = (-fB+sqrt(fDesc))/(2*fA);
		fX2 = (-fB-sqrt(fDesc))/(2*fA);
		printf("\nThe Roots are Real and distinct, they are \n");
		printf("\nRoot1 = %g and Root2 = %g\n",fX1,fX2);
	}
	else
	{
		fRealp = -fB / (2*fA);
		fImagp = sqrt(fabs(fDesc))/(2*fA);
		printf("\nThe Roots are imaginary and they are\n");
		printf("\nRoot1 = %g+i%g\n",fRealp,fImagp);
		printf("\nRoot2 = %g-i%g\n",fRealp,fImagp);
	}

	return 0;
}

\end{Verbatim}


\section*{Output}
Run the following commands in your terminal:\\
\textbf{\$ gcc 01Quadratic.c -lm \\ \$./a.out }
\begin{Verbatim}
*************************************************************
*       PROGRAM TO FIND ROOTS OF A QUADRATIC EQUATION       *
*************************************************************
Enter the coefficients of a,b,c 
0 1 2

Invalid input, not a quadratic equation - try again
\end{Verbatim}
\textbf{\$./a.out }
\begin{Verbatim}
*************************************************************
*       PROGRAM TO FIND ROOTS OF A QUADRATIC EQUATION       *
*************************************************************
Enter the coefficients of a,b,c 
1 -5 6

The Roots are Real and distinct, they are 

Root1 = 3 and Root2 = 2
\end{Verbatim}
\textbf{\$./a.out }
\begin{Verbatim}
*************************************************************
*       PROGRAM TO FIND ROOTS OF A QUADRATIC EQUATION       *
*************************************************************
Enter the coefficients of a,b,c 
1 4 4

Roots are equal and the Roots are 

Root1 = -2 and Root2 = -2
\end{Verbatim}
\textbf{\$./a.out }
\begin{Verbatim}
*************************************************************
*       PROGRAM TO FIND ROOTS OF A QUADRATIC EQUATION       *
*************************************************************
Enter the coefficients of a,b,c 
1 3 3

The Roots are imaginary and they are

Root1 = -1.5+i0.866025
Root2 = -1.5-i0.866025
\end{Verbatim}

\chapter{Palindrome}
\section*{Summary}
\section*{Algorithm}
\section*{C Code}
\section*{Output}

\chapter{Square Root}
\section*{Summary}
\section*{Algorithm}
\section*{C Code}
\section*{Output}

\chapter{Leap Year Check}
\section*{Summary}
\section*{Algorithm}
\section*{C Code}
\section*{Output}
\end{document}
